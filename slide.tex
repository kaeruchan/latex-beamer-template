\documentclass{beamer}
\usetheme{Gelugor}
\usepackage[utf8]{inputenc}
\usepackage[T1]{fontenc}
\usepackage{newtxtext,newtxmath}
%% Use any fonts you like.
\usepackage{helvet}
\usepackage{graphicx}	
\usepackage{amsmath,bbm}
\usepackage[draft]{pdfcomment}


% reference package
%\usepackage[style=ieee]{biblatex} % If you want use bibtex, you can use \footfullcite{Bibtexkey} command
%\addbibresource{citation.bib} % add your bibtex library here
%\setbeamerfont{footnote}{size=\tiny}

\title{\large \textcolor{white}{Template for YNU}}
\subtitle{Template for Ochiailab}
\author{{\bf Author:}Tianji Shen~~~~{\bf Supervisor:}Prof. Hideki Ochiai}
\date{Feb. 15, 2018}
\institute{Dept. of Electrical and Computing Engineering\\
Yokohama National University}



\begin{document}

\begin{frame}[plain,t]
\titlepage
\end{frame}

%outline
\begin{frame}{Table of contents}
  \tableofcontents
\end{frame}



\begin{frame}{Background}
 \frametitle{A Frame}
 \framesubtitle{Bullet points}
 \begin{itemize}
  \item First thing
        \begin{itemize}
         \item small point
         \item fine print
        \end{itemize}
  \item Second thing
        \begin{enumerate}
         \item point 1
        \end{enumerate}
  \item Third thing
        \begin{description}
         \item[Research] the scientific pursuit for knowledge
        \end{description}
 \end{itemize}
\end{frame}

\subsection{Text}
\begin{frame}
\frametitle{Another Frame}
 Test for a new frame.
 \end{frame}

 \subsection{Blocks}
 \begin{frame}
 \frametitle{Blocks}
 \begin{definition}[Greetings]
 Hello World
 \end{definition}

  \begin{theorem}[Fermat's Last Theorem]
   $a^n + b^n = c^n, n \leq 2$
  \end{theorem}

  \begin{alertblock}{Uh-oh.}
   By the pricking of my thumbs.
  \end{alertblock}

  \begin{exampleblock}{Uh-oh.}
   Something evil this way comes.
  \end{exampleblock}

 \end{frame}

\ThankYouFrame




    % \bibliographystyle{IEEEtran}
    % \bibliography{citation}

\end{document}